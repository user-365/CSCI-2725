
% Template.tex's last substantial revision: 2023-03-03 16:37 EST
% "Substantial" for Template.tex is laxer than for content-ful .texs
% Revision contents: add \del macro


%%%%%%%%%%%%%%%%%%%%%%%%%%%%%%%%%%%%%%%%%%%%%%%%%%%%%%%%%%%%%%%%%%%%%%%%%%%%%%%
% MANAGE PDFs FOR SCREEN-READING %%%%%%%%%%%%%%%%%%%%%%%%%%%%%%%%%%%%%%%%%%%%%%
%%%%%%%%%%%%%%%%%%%%%%%%%%%%%%%%%%%%%%%%%%%%%%%%%%%%%%%%%%%%%%%%%%%%%%%%%%%%%%%
%\RequirePackage{pdfmanagement-testphase} % TK Uncomment when dependencies gotten
%\DeclareDocumentMetadata{uncompress,pdfversion=2.0} % TK uncomment when control sequence done



%%%%%%%%%%%%%%%%%%%%%%%%%%%%%%%%%%%%%%%%%%%%%%%%%%%%%%%%%%%%%%%%%%%%%%%%%%%%%%%
% TK LINKS (INSPIRATION) %%%%%%%%%%%%%%%%%%%%%%%%%%%%%%%%%%%%%%%%%%%%%%%%%%%%%%
%%%%%%%%%%%%%%%%%%%%%%%%%%%%%%%%%%%%%%%%%%%%%%%%%%%%%%%%%%%%%%%%%%%%%%%%%%%%%%%
% inspo for templating: https://tex.stackexchange.com/q/59702
% for header/footer: https://tex.stackexchange.com/a/203059



%%%%%%%%%%%%%%%%%%%%%%%%%%%%%%%%%%%%%%%%%%%%%%%%%%%%%%%%%%%%%%%%%%%%%%%%%%%%%%%
% TK LINKS (GUIDES) %%%%%%%%%%%%%%%%%%%%%%%%%%%%%%%%%%%%%%%%%%%%%%%%%%%%%%%%%%%
%%%%%%%%%%%%%%%%%%%%%%%%%%%%%%%%%%%%%%%%%%%%%%%%%%%%%%%%%%%%%%%%%%%%%%%%%%%%%%%
% Typography in general: https://practicaltypography.com/summary-of-key-rules.html
% LaTeX mistakes (see also comments): http://www.johndcook.com/blog/2010/02/15/top-latex-mistakes/



%%%%%%%%%%%%%%%%%%%%%%%%%%%%%%%%%%%%%%%%%%%%%%%%%%%%%%%%%%%%%%%%%%%%%%%%%%%%%%%
% TK TODO %%%%%%%%%%%%%%%%%%%%%%%%%%%%%%%%%%%%%%%%%%%%%%%%%%%%%%%%%%%%%%%%%%%%%
%%%%%%%%%%%%%%%%%%%%%%%%%%%%%%%%%%%%%%%%%%%%%%%%%%%%%%%%%%%%%%%%%%%%%%%%%%%%%%%
%
% Document-level
% - make separate latex file to catalogue (my) custom macros
% - make separate latex file as my design ethos manifesto
%
% Component-level
% - showcase other things (e.g. codeblocks, highlighting, qeding)
% - make remarks, proof? blocks (tcolorbox)
% - add captions/labels to mymintedcodeblocks (tcolorbox)
%
% Text-level
% - fix vertical spacing between Comment and math
% - directional (from NW, bend, to W) arrows for proof comments (first instance)
% - fix square root sign stroke thickness (too thick) in single-$/text-math mode
% ? align 2nd level enumeration items with 1st level (ask on stackexchange)
% ? center equations in prompt in various problems
%
% Comment-level
% - switch "See" to "Credit" appropriately



%%%%%%%%%%%%%%%%%%%%%%%%%%%%%%%%%%%%%%%%%%%%%%%%%%%%%%%%%%%%%%%%%%%%%%%%%%%%%%%
% CLASS %%%%%%%%%%%%%%%%%%%%%%%%%%%%%%%%%%%%%%%%%%%%%%%%%%%%%%%%%%%%%%%%%%%%%%%
%%%%%%%%%%%%%%%%%%%%%%%%%%%%%%%%%%%%%%%%%%%%%%%%%%%%%%%%%%%%%%%%%%%%%%%%%%%%%%%
\documentclass[12pt, a4paper]{article}



%%%%%%%%%%%%%%%%%%%%%%%%%%%%%%%%%%%%%%%%%%%%%%%%%%%%%%%%%%%%%%%%%%%%%%%%%%%%%%%
% FONTS and ENCODING %%%%%%%%%%%%%%%%%%%%%%%%%%%%%%%%%%%%%%%%%%%%%%%%%%%%%%%%%%
%%%%%%%%%%%%%%%%%%%%%%%%%%%%%%%%%%%%%%%%%%%%%%%%%%%%%%%%%%%%%%%%%%%%%%%%%%%%%%%
%
% See: https://tex.stackexchange.com/q/59702
\usepackage[utf8]{inputenc}	      % allows input of accented characters directly from the keyboard
							              			% https://tex.stackexchange.com/questions/44694/fontenc-vs-inputenc
\usepackage[T1]{fontenc} 	        % ensures proper hyphenation
%
\usepackage{libertine}						% Linux Libertine (text) font family
\usepackage[libertine,						% math-mode fonts to harmonize with Libertine text fonts.
smallerops, varg, cmbraces, noamssymbols, ]{newtxmath}
                                  % [libertine] = substitutes libertine for Times
                                  % [smallerops] = smaller \sum,etc.
                                  % [varg] = distinctive math letters
                                  % [cmbraces] = thicker braces
                                  % [noamssymbols] = prevent TXAMSadd from loading
\let\Bbbk\relax										% disables newtxmath's definition of \Bbbk in favor of amssymb's (blackboard bold lowercase k)
                                  % This choice of disabling is arbitrary.
                                  % See: https://tex.stackexchange.com/a/587033
%\usepackage{newtxtext}		  		  % loads clones of reasonably well-match fonts (Helvetica and a monospaced)
%\usepackage{newpxtext,newpxmath} % based on Zapf's Palatino font
%
\usepackage{microtype}						% to break monospace (texttt) lines correctly
% the below does not work with ragged-right env. (leave disabled to stop warnings)
%\usepackage[htt]{hyphenat}       % to hyphenate monospace (texttt) lines correctly
							              			% See: https://tex.stackexchange.com/a/10378



%%%%%%%%%%%%%%%%%%%%%%%%%%%%%%%%%%%%%%%%%%%%%%%%%%%%%%%%%%%%%%%%%%%%%%%%%%%%%%%
% AMS MATH %%%%%%%%%%%%%%%%%%%%%%%%%%%%%%%%%%%%%%%%%%%%%%%%%%%%%%%%%%%%%%%%%%%%
%%%%%%%%%%%%%%%%%%%%%%%%%%%%%%%%%%%%%%%%%%%%%%%%%%%%%%%%%%%%%%%%%%%%%%%%%%%%%%%
% \usepackage{amsmath}      % loads amstext, amsbsy, amsopn but not amssymb
                            % equation stuff (eqref, subequations, equation, align, gather, flalign, multline, alignat, split...)
                            % redundant by newtxmath
% \usepackage{amsfonts}     % may be redundant with amsmath
\usepackage{amssymb}        % may be redundant with amsmath (enable for \therefore)
% \numberwithin{equation}{section}  % reset equation counters at start of each "section" and prefix numbers by section number
% \numberwithin{figure}{section}    % reset figure   counters at start of each "section" and prefix numbers by section number



%%%%%%%%%%%%%%%%%%%%%%%%%%%%%%%%%%%%%%%%%%%%%%%%%%%%%%%%%%%%%%%%%%%%%%%%%%%%%%%
% LAYOUT %%%%%%%%%%%%%%%%%%%%%%%%%%%%%%%%%%%%%%%%%%%%%%%%%%%%%%%%%%%%%%%%%%%%%%
%%%%%%%%%%%%%%%%%%%%%%%%%%%%%%%%%%%%%%%%%%%%%%%%%%%%%%%%%%%%%%%%%%%%%%%%%%%%%%%
%
% force ~71 (65) characters per line
% Credit: https://tex.stackexchange.com/a/59636
\usepackage{geometry}
\newlength{\alphabet}
\settowidth{\alphabet}{\normalfont abcdefghijklmnopqrstuvwxyz}
\geometry{textwidth = 2.75\alphabet}		% (Note: 2.75 * 26 = 71.5)
% wider for math-heavy, narrower for text-heavy
\makeatletter
\if@twoside
  \geometry{hmarginratio = {2:3}}
\else
  \geometry{hmarginratio = {1:1}}
\fi
\makeatother
%
%
\linespread{1.25}
%
% remove spaces around align (and other display math) envs
% Credit: https://tex.stackexchange.com/a/47403
%\usepackage{etoolbox}
%\newcommand{\zerodisplayskips}{%
%  \setlength{\abovedisplayskip}{12pt}%
%  \setlength{\belowdisplayskip}{12pt}%
%% "short skip" refers to when the text line above/below the display math
%% --ENDS (horizontally) before the display math BEGINS (horizontally)
%% --so the math is raised a bit (vertically) to not leave so much whitespace.
%% --See: https://tex.stackexchange.com/a/30913
%  \setlength{\abovedisplayshortskip}{5pt}%
%  \setlength{\belowdisplayshortskip}{5pt}}
%\appto{\normalsize}{\zerodisplayskips}
%\appto{\small}{\zerodisplayskips}
%\appto{\footnotesize}{\zerodisplayskips}
%
%
\usepackage{enumitem}
\usepackage{mathtools}				% for multlined, dcases environment
\usepackage{lipsum}						% to fill in with arbitrary text
%\usepackage{datetime2}				% to display \currenttime in \date
%
\usepackage{ntheorem}
\newtheorem{theorem}{Theorem}
%
\usepackage{tcolorbox}				% currently, only for CUSTOM RENDERED CODE BLOCKS
\tcbuselibrary{most}					% currently, only for myhighlight
\usepackage{fancybox}         % for text boxes such as those surrounding \Comments
%
\usepackage{derivative}			  % for upright "d"s (!) and all sorts of derivatives
%
% redefine \int to have (on top of default) negative space (idk how much (mathopen{}))
% works with varargs bounds/limits
% Credit: https://tex.stackexchange.com/a/277428
\makeatletter
\let\int@original\int
\def\int{\int@checkfirstsb}
\def\int@checkfirstsb{\@ifnextchar_{\int@checksecondsp}{\int@checkfirstsp}}
\def\int@checkfirstsp{\@ifnextchar^{\int@checksecondsb}{\int@{}{}}}
\def\int@checksecondsp_#1{\@ifnextchar^{\int@grabsp{#1}}{\int@{#1}{}}}
\def\int@checksecondsb^#1{\@ifnextchar_{\int@grabsb{#1}}{\int@{}{#1}}}
\def\int@grabsb#1_#2{\int@{#2}{#1}}
\def\int@grabsp#1^#2{\int@{#1}{#2}}
\def\int@#1#2{\int@original\ifblank{#1}{}{_{#1}}\ifblank{#2}{}{^{#2}}\mathopen{}} % redefinition
\makeatother



%%%%%%%%%%%%%%%%%%%%%%%%%%%%%%%%%%%%%%%%%%%%%%%%%%%%%%%%%%%%%%%%%%%%%%%%%%%%%%%
% GRAPHICX %%%%%%%%%%%%%%%%%%%%%%%%%%%%%%%%%%%%%%%%%%%%%%%%%%%%%%%%%%%%%%%%%%%%
%%%%%%%%%%%%%%%%%%%%%%%%%%%%%%%%%%%%%%%%%%%%%%%%%%%%%%%%%%%%%%%%%%%%%%%%%%%%%%%
\usepackage{graphicx} 		            % options = [final]  = all graphics displayed, regardless of draft option in class
                                      % ^[final] does option clash
                                      % options = [pdftex] = necessary (?) if import PDF files
                                      % no option : when importing ps- and eps-files (?)
\usepackage[export]{adjustbox}
\graphicspath{ {./images/} }  	  		% tell LaTeX where to look for images
% \DeclareGraphicsExtensions{.pdf, .PDF, .jpg, .JPG, .jpeg, .JPEG, .png, .PNG, .bmp, .BMP, .eps, .ps}
\usepackage{float}				            % Improved interface for floating objects; add [H] option
\usepackage{subcaption}		            % allows subfigures (inside figure envs)



%%%%%%%%%%%%%%%%%%%%%%%%%%%%%%%%%%%%%%%%%%%%%%%%%%%%%%%%%%%%%%%%%%%%%%%%%%%%%%%
% HYPERREF (last) then HYPCAP %%%%%%%%%%%%%%%%%%%%%%%%%%%%%%%%%%%%%%%%%%%%%%%%%
%%%%%%%%%%%%%%%%%%%%%%%%%%%%%%%%%%%%%%%%%%%%%%%%%%%%%%%%%%%%%%%%%%%%%%%%%%%%%%%
%
% See: https://tex.stackexchange.com/q/1863
\usepackage[
  pdflang={en-US},
  pdftex, 
  final,                      % if you DO want to have clickable-colorful links
  pdfstartview = FitV,
  linktocpage  = false,       % ToC, LoF, LoT place hyperlink on page number, rather than entry text
  breaklinks   = true,        % so long urls are correctly broken across lines
  % pagebackref  = false,     % add page number in bibliography and link to position in document where cited
]{hyperref}					          % hyperref should be last loaded package; see link above
%
% make better PDF
\hypersetup{
  pdftitle  = {Penny-Wisdom in \LaTeX},
  pdfauthor = {Yitao Tian},
} % setup PDF
%\usepackage{tagpdf} % TK uncomment. add tags to content



%%%%%%%%%%%%%%%%%%%%%%%%%%%%%%%%%%%%%%%%%%%%%%%%%%%%%%%%%%%%%%%%%%%%%%%%%%%%%%%
% CUSTOM MACROS %%%%%%%%%%%%%%%%%%%%%%%%%%%%%%%%%%%%%%%%%%%%%%%%%%%%%%%%%%%%%%%
%%%%%%%%%%%%%%%%%%%%%%%%%%%%%%%%%%%%%%%%%%%%%%%%%%%%%%%%%%%%%%%%%%%%%%%%%%%%%%%
% right align qed symbol (for non-proof-environments)
\newcommand{\QED}{\tag*{$\blacksquare$}}
%
% creates a phantom relational symbol (e.g. "=" (default) or "\leq")
% multiline/split expressions (after relational) must be horizontally to the right of the relational
% Credit: https://tex.stackexchange.com/a/448386
\newcommand*{\PhantomRel}[1][=]{\mathrel{\phantom{#1}}}
%
% word/phrase inline highlighting (for final answers). Color: green
% Credit: https://tex.stackexchange.com/a/234175 (less SLOW than tikz method on same page?)
\newtcbox{\Highlight}[1][]{enhanced, colframe = green, colback = green!15, 
       frame style = {opacity = 0.25}, interior style = {opacity = 0.25}, 
       nobeforeafter, tcbox raise base, shrink tight, extrude by = 1mm, #1}
%
% comment to explain step of proof
% Catalogue of boxes (e.g. ovalbox): https://tex.stackexchange.com/a/373420
\newcommand*{\Comment}[1]{\tag*{\ovalbox{\begin{tiny}%
  \textrm{#1}%
\end{tiny}}}}
%
% repackaging of BArr env below so it's probably redundant
\newcommand{\ArrVect}[1]{\begin{BArr}#1\end{BArr}}
%
% bold-roman to represent vectors (as variables)
\newcommand*{\Vect}[1]{\mathbf{#1}}
%
% Credit: https://tex.stackexchange.com/a/15897
% #1: expression to be evaluated;
% #2: variable to be set; #3: lower bound; #4: upper bound
\newcommand{\Eval}[4]{\left.#1\mspace{2mu}\right\rvert_{\,#2\,=\,#3}^{\,#4}}
%
% "such that" bar in set-builder notation
\newcommand{\SuchThat}{\mathrel{}\middle|\mathrel{}}
%
% shorthand for \odif of `derivative` package
% varargs (csv) to match \odif's csv varargs
% See (csv varargs): https://tex.stackexchange.com/a/286717
\renewcommand*{\d}[1]{%
  \forcsvlist\odif{#1}%
}
%
% "del" operator for boundaries of spaces, from the `derivative` package
% of note: "style-notation=multiple" makes the variable (not "del") NOT SUBscript
% command's naming follows that of `derivative` package
\newcommand*{\del}[1]{\pdif[style-notation=multiple][#1]}
%
% (hyper-)volume differential, but "Vol" is romanized (unlike x or y (italicized))
\newcommand*{\dVol}{\odif{\textrm{Vol}}}
%
% Credit: https://tex.stackexchange.com/a/43009
% absolute value and norm delimiters
\DeclarePairedDelimiter\abs{\lvert}{\rvert}%
\DeclarePairedDelimiter\norm{\lVert}{\rVert}%
% Swap the definitions so that \abs,\norm resizes the brackets,
% and the starred (*) versions do not.
\makeatletter
\let\oldabs\abs
\def\abs{\@ifstar{\oldabs}{\oldabs*}}
\let\oldnorm\norm
\def\norm{\@ifstar{\oldnorm}{\oldnorm*}}
\makeatother
%
% R^n (isomorphic to) euclidean space.
% #1: number of dimensions
\newcommand*{\R}[1][n]{\mathbf{R}^{#1}}



%%%%%%%%%%%%%%%%%%%%%%%%%%%%%%%%%%%%%%%%%%%%%%%%%%%%%%%%%%%%%%%%%%%%%%%%%%%%%%%
% CUSTOM ENVIRONMENTS %%%%%%%%%%%%%%%%%%%%%%%%%%%%%%%%%%%%%%%%%%%%%%%%%%%%%%%%%
%%%%%%%%%%%%%%%%%%%%%%%%%%%%%%%%%%%%%%%%%%%%%%%%%%%%%%%%%%%%%%%%%%%%%%%%%%%%%%%
%
% Shifrin problem enumerate
\newlist{Shifrin}{enumerate}{2}         % Custom 2-deep enumerate list
\setlist[Shifrin, 1]{%                  % Configure the behaviour of level 1 entries
    label				= \arabic{Shifrini}.,   % 1., 2., 3., ...
    leftmargin	= \parindent,
    rightmargin	= 10pt
}
\setlist[Shifrin, 2]{%                  % Configure the behaviour of level 2 entries
    label				= (\arabic{Shifrinii}), % e.g. (1), (2), (3), ...
    leftmargin	= 15pt,
    rightmargin	= 15pt
}
%
% Bracketed array (i.e., column/row vector, matrix)
% Credit: https://tex.stackexchange.com/a/156022
\newenvironment{BArr}{%         % short for "Bracketed Array"
  \!\!\:\Bigl[\!\begin{smallmatrix}}{%
  \end{smallmatrix}\!\Bigr]}
%

% REPLACE WITH ALL OF ABOVE
\usepackage{MintedUnbreakableCodeBlock}



%%%%%%%%%%%%%%%%%%%%%%%%%%%%%%%%%%%%%%%%%%%%%%%%%%%%%%%%%%%%%%%%%%%%%%%%%%%%%%%
% TITLE PAGE %%%%%%%%%%%%%%%%%%%%%%%%%%%%%%%%%%%%%%%%%%%%%%%%%%%%%%%%%%%%%%%%%%
%%%%%%%%%%%%%%%%%%%%%%%%%%%%%%%%%%%%%%%%%%%%%%%%%%%%%%%%%%%%%%%%%%%%%%%%%%%%%%%
\title{CSCI-2725 Homework 2}
\author{Yitao Tian}
\date{\today}



%%%%%%%%%%%%%%%%%%%%%%%%%%%%%%%%%%%%%%%%%%%%%%%%%%%%%%%%%%%%%%%%%%%%%%%%%%%%%%%
% THE ACTUAL DOCUMENT %%%%%%%%%%%%%%%%%%%%%%%%%%%%%%%%%%%%%%%%%%%%%%%%%%%%%%%%%
%%%%%%%%%%%%%%%%%%%%%%%%%%%%%%%%%%%%%%%%%%%%%%%%%%%%%%%%%%%%%%%%%%%%%%%%%%%%%%%
\begin{document}

\maketitle

\textbf{Extra credit:} There is 5 percentage extra credit if you don't submit hand-written homework.
You can use \LaTeX or any other tool to write your homework.

\begin{enumerate}
  \item (5 points each) Look at the pseudocode below,
  and find the Big-$O$ estimate or worst case complexity of the following recursive functions.

  \textbf{Hint:} Write recurrence relation for the functions and solve it using Master's Theorem.
\end{enumerate}

\begin{MintedUnbreakableCodeBlock}{java}
      void fun(int n, int m) {        // $T(l)=$
        if (m > n) return;
        System.out.print("m = " + m); // $1+$
        fun(n, m + 2);                // $T(l-2)$
      }
\end{MintedUnbreakableCodeBlock}

Thus, the recurrence relation for the above function is
\begin{equation*}
  T(l) = 1 + T(l-2) (\textrm{letting } l \coloneqq m-n-1), \quad T(0) = 1.
\end{equation*}

Fitting this to the hypothesis of the \textit{Subtract and Conquer} version of the Master Theorem,
we have $a=1$, $b=2$, and $f(l)\in O(1)=O(l^d)$, with $d=0$.

By the Subtract and Conquer version, we have that:
\begin{equation*}
  \textrm{Since } a = 1, \textrm{ then } \Highlight{$T(l)\in O(l^{d+1}) = O(l)$}. \QED
\end{equation*}



\newpage



\begin{MintedUnbreakableCodeBlock}{java}
      float fun(int a[], int i) {                    // $T(i)=$
        if (i == 0) {
          return a[0];
        }
        if (i > 0) {
          return (i * fun(a, i - 1) + a[i])/(i + 1); // $T(i-1)$
        }
      }
\end{MintedUnbreakableCodeBlock}

Thus, the recurrence relation for the above function is
\begin{equation*}
  T(i) = T(i-1), \quad T(0) = 1.
\end{equation*}

Putting this into the hypothesis of the S\&C version,
we have $a=1$, $b=1$, and $f(i)\in O(i^d)$, with $d=0$.

Applying the S\&C version, we have that
\begin{equation*}
  \textrm{Since } a = 1, \textrm{ then } \Highlight{$T(i)\in O(i^{d+1}) = O(i)$}. \QED
\end{equation*}



\newpage



Assume that the function "random" has a complexity of $O(n)$.
\begin{MintedUnbreakableCodeBlock}{java}
      void fun(int array[], int a, int b) { // $T(n)=$
        if (a > b) return;
        mid = (a + b)/2;                    // $1+$
        fun(array, a, mid);                 // $T(n/2)+$
        fun(array, mid + 1, b);             // $T(n/2)+$
        random(array, a, mid, b);           // $g(n)$
      }
\end{MintedUnbreakableCodeBlock}

Thus, the recurrence relation for the above is
\begin{equation*}
  T(n) = 2T(n/2) + f(n), T(0) = 1,
\end{equation*}
(letting $n \coloneqq a-b-1$ and $f(n) \coloneqq 1+g(n)$, with $g(n)\in O(n)$).

We cannot use the S\&C version of the Master Theorem,
but instead the \textit{Divide and Conquer} version.
Putting it into the hypothesis of the D\&C,
we have that $a=2$, $b=2$, and $f(n)\in\Theta(n^1) = \Theta(n^{\log_b(a)})$.

Applying the D\&C, we have
\begin{equation*}
  \textrm{Since } f(n)\in\Theta(n^1) = \Theta(n^{\log_ba}), \Highlight{$T(n)\in\Theta(n \log n)$}. \QED
\end{equation*}



\newpage



\begin{MintedUnbreakableCodeBlock}{java}
      int fun(int x, int n) {                     // $T(n)=$
        if (n == 0) return 1;
        if (x == 0) return 0;
        if (n == 1) return x;
        if (n % 2 == 0) return fun(x * x, n / 2);
        else return x * fun(x * x, n / 2);        // $T(n/2)$
      }
\end{MintedUnbreakableCodeBlock}

Thus, the recurrence relation for the above function is
\begin{equation*}
  T(n) = T(n/2), \quad T(0) = 1.
\end{equation*}

In terms of the hypothesis of the D\&C version,
we have $a=1$, $b=2$, and $f(n)\in O(1) = O(n^{\log_ba - \varepsilon})$, for all $\varepsilon=0$.

Applying the D\&C, we have
\begin{equation*}
  \textrm{Since } f(n)\in O(n^{\log_ba - \varepsilon}), \textrm{ then } \Highlight{$T(n)\in\Theta(n^0)$}.
\end{equation*}



\newpage



\begin{enumerate}[resume]

  \item (10 points) Consider a recurrence relation given by following.
  \begin{equation*}
    T(n) = 2T(n-1) + 2^n
  \end{equation*}
  
  \textbf{Show} that $T(n)=n2^n$ is the solution of the above equation.
  If you need a base case, use $T(0)=0$.

  \textbf{Answer:} Using backward substitution, we have the following:
  \begin{align*}
    T(n)
      &= 2T(n-1) + 2^n \\
      &= 2(2T(n-2) + 2^n) + 2^n \\
      &= 2(2(2T(n-3) + 2^n) + 2^n) + 2^n \\
      \vdots
  \end{align*}

  Undertaking a proof by induction, let $k$ be the numbering of $P$s as follows:
  \begin{enumerate}
    
    \item $P(0)$ is the statement that $T(n)$ equals an expression in terms of $T(n-1)$,
      namely, $T(n) = 2T(n-1) + 2^n$;

    \item $P(k)$ is the statement that $T(n)$ equals some expression in terms of $T(n-1-k)$,
     namely, $T(n) = 2(2(\dots 2T(n-1-k) + 2^n \dots) + 2^n) + 2^n$;

    \item and of course, $P(k+1)$ is one in terms of $T(n-1-(k+1))$.

  \end{enumerate}

  From the recurrence relation derived from the source code above, $P(0)$ follows.

  Since $T(n)$ represents a recursion, we have that:
  For all $n_0\in [0,n]$, $T(n_0) = 2T(n_0-1)+2^n$.
  Therefore, $T(n_0) = 2T(n_0-1)+2^n$.
  Noting the obvious, if $n_0=n-1-k$, then $n_0-1=n-1-(k+1)$.
  $P(k)\rightarrow P(k+1)$ follows immediately.

  Thus, by induction, we have that $P(k)$ is true for all $k\in [0,n-1]$.

  From the definition of $P(k)$, we have
  \begin{align*}
    T(n)
      &= 2(2(\dots 2T(n-1-a) + 2^n \dots) + 2^n) + 2^n \\
      &= 2^a T(n-1-a) + \underbrace{2^{n+a-1} + 2^{n+a-2} + \cdots + 2^{n+(a-(a-2))-1} + 2^{n+(a-(a-1))-1}}_a \\
      &= 2^{n-1} T(0) + \underbrace{2^{n+(n-1)-1} + 2^{n+(n-1)-2} + \cdots + 2^{n+((n-1)-(n-1-1))-1}}_{n-1}
        & \Comment{let $a=n-1$} \\
      &= \underbrace{2^{2n-2} + 2^{2n-3} + \cdots + 2^{2n-n}}_{n-1}
        & \Comment{$T(0)=0$} \\
      &= \sum_{i=0}^{n-2} 2^{2n-2} 2^{-i}.
  \end{align*}

  The sum is that of a geometric progression, with $a=2^{n-2}$ and $r=0.5$.
  Generally, the sum of a geometric progression up to the $k$-th term is given as:
  \begin{equation*}
    S_k = \frac{ar^{k+1} - a}{r - 1}.
  \end{equation*}

  Here, $k=n-2$, so
  \begin{align*}
    \sum_{i=0}^{n-2} 2^{2n-2} 2^{-i}
      &= \frac{2^{2n-2}(0.5)^{n-2+1} - 2^{2n-2}}{0.5 - 1} \\
      &= \frac{2^{2n-2} - 2^{2n-2}2^{-n+1}}{0.5} \\
      &= \frac{2^{n-1} \left(2^{n-1} - 1\right)}{0.5}
  \end{align*}



  \newpage



  \item (10 points) Below is the code for Fibonacci sequence. \textbf{Write} a recurrence relation for this code and solve it using substitution (forward or backward) method.
  
  \textbf{Note:} There are two recurrence relations $T(n-1)$ and $T(n-2)$ and to simplify your calculation you can assume that $T(n-1)=T(n-2)$ as they are almost of same size.

\end{enumerate}

\begin{MintedUnbreakableCodeBlock}{java}
      int Fibonacci(int N) {                          // $T(n)=$
        if (n == 0 || n == 1) {
          return 1;
        }
        else {
          return Fibonacci(n - 1) + Fibonacci(n - 2); // $T(n-1)+T(n-2)$
        }
      }
\end{MintedUnbreakableCodeBlock}

Since we are given that $T(n-1) \approx T(n-2)$,
the recurrence relation for the above then becomes
\begin{equation*}
  T(n) = 2T(n-1), \quad T(0) = 1.
\end{equation*}

\textbf{Answer:} Using forward substitution, we have the following:
\begin{align*}
  T(1) &= 2T(0) \\
  \therefore T(2) &= 2(2T(0)) \\
  &\vdots \\
  T(n) &= 2^n T(0) \\
  \therefore T(n) &= \Highlight{$2^n$}.
    & \Comment{$T(0)=1$}
\end{align*}



\end{document}