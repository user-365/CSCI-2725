% Homework-3.tex's last substantial revision: 2023-03-18 10:50 EST
% Template.tex's last substantial revision: 2023-03-09 10:29 EST
% "Substantial" for Template.tex is laxer than for content-ful .texs
% Revision contents: add white/weaker QED


%%%%%%%%%%%%%%%%%%%%%%%%%%%%%%%%%%%%%%%%%%%%%%%%%%%%%%%%%%%%%%%%%%%%%%%%%%%%%%%%
% MANAGE PDFs FOR SCREEN-READING %%%%%%%%%%%%%%%%%%%%%%%%%%%%%%%%%%%%%%%%%%%%%%%
%%%%%%%%%%%%%%%%%%%%%%%%%%%%%%%%%%%%%%%%%%%%%%%%%%%%%%%%%%%%%%%%%%%%%%%%%%%%%%%%
%\RequirePackage{pdfmanagement-testphase} % TK Uncomment when dependencies gotten
%\DeclareDocumentMetadata{uncompress,pdfversion=2.0} % TK uncomment when control sequence done
% See: TK idk find the url



%%%%%%%%%%%%%%%%%%%%%%%%%%%%%%%%%%%%%%%%%%%%%%%%%%%%%%%%%%%%%%%%%%%%%%%%%%%%%%%%
% TK LINKS (INSPIRATION) %%%%%%%%%%%%%%%%%%%%%%%%%%%%%%%%%%%%%%%%%%%%%%
%%%%%%%%%%%%%%%%%%%%%%%%%%%%%%%%%%%%%%%%%%%%%%%%%%%%%%%%%%%%%%%%%%%%%%%%%%%%%%%%
% `standalone` package: put pictures etc. into own source files
% for header/footer: https://tex.stackexchange.com/a/203059
% for info about draft mode: https://tex.stackexchange.com/q/49277



%%%%%%%%%%%%%%%%%%%%%%%%%%%%%%%%%%%%%%%%%%%%%%%%%%%%%%%%%%%%%%%%%%%%%%%%%%%%%%%%
% TK LINKS (GUIDES) %%%%%%%%%%%%%%%%%%%%%%%%%%%%%%%%%%%%%%%%%%%%%%%%%%%%%%%%%%%%
%%%%%%%%%%%%%%%%%%%%%%%%%%%%%%%%%%%%%%%%%%%%%%%%%%%%%%%%%%%%%%%%%%%%%%%%%%%%%%%%
% Proofwiki House Style: https://proofwiki.org/wiki/Help:Editing/House_Style
% Typography in general: https://practicaltypography.com/summary-of-key-rules.html
% LaTeX mistakes (see also comments): http://www.johndcook.com/blog/2010/02/15/top-latex-mistakes/
% best practices for templating: https://tex.stackexchange.com/questions/40760/best-practice-on-organising-your-preamble
% .dtx tutorial: https://www.ctan.org/tex-archive/info/dtxtut
% standalone and .fmts: https://tex.stackexchange.com/questions/42414/using-the-standalone-package-with-a-fmt-file
% why put title+metadata before hyperref?: https://tex.stackexchange.com/questions/17218/make-hyperref-take-pdfinfo-from-title-and-author
% for fixing vscode/latex workshop pdf viewer: https://tex.stackexchange.com/questions/527463/pdf-preview-in-visual-studio-code
% for CV: https://latex-tutorial.com/cv-latex-guide/
% inspo for templating: https://tex.stackexchange.com/q/59702
% for making captions of minteds: https://tex.stackexchange.com/a/305392
% for speeding up TikZ: https://tex.stackexchange.com/q/45
% for improving minteds: https://latex-tutorial.com/code-listings/
% for highlighting code (colored boxes) on top of syntax highlighting: https://tex.stackexchange.com/a/49309



%%%%%%%%%%%%%%%%%%%%%%%%%%%%%%%%%%%%%%%%%%%%%%%%%%%%%%%%%%%%%%%%%%%%%%%%%%%%%%%%
% TK TODO %%%%%%%%%%%%%%%%%%%%%%%%%%%%%%%%%%%%%%%%%%%%%%%%%%%%%%%%%%%%%%%%%%%%%%
%%%%%%%%%%%%%%%%%%%%%%%%%%%%%%%%%%%%%%%%%%%%%%%%%%%%%%%%%%%%%%%%%%%%%%%%%%%%%%%%
%
% Document-level
% - make separate latex file to preload preamble
% - make separate latex file as my design ethos manifesto
%
% Component-level
% - showcase other things (e.g. codeblocks, highlighting, qeding)
% - make remarks and proof? blocks (tcolorbox)
%
% Text-level
% - override \Vec (`asmmath`?) command with my own
% - fix vertical spacing between Comment and math
% - directional (from NW, bend, to W) arrows for proof comments (first instance)
% ✅ align 2nd level enumeration items with 1st level (ask on stackexchange)
% --- create "CSCI-2725" enum environment to handle this
% --- solution (in question): https://tex.stackexchange.com/q/419352/
% - fix square root sign stroke thickness (too thick) in single-$/text-math mode
% ✅ center (align*) equations in prompt in various problems (enumerate)
% --- solution (2nd part of ans.): https://tex.stackexchange.com/a/54686/
% --- other solution (more flexible but needs multiple compilation): https://tex.stackexchange.com/a/9640/
%
% Comment-level
% - switch "See" to "Credit" appropriately



%%%%%%%%%%%%%%%%%%%%%%%%%%%%%%%%%%%%%%%%%%%%%%%%%%%%%%%%%%%%%%%%%%%%%%%%%%%%%%%%
% CLASS %%%%%%%%%%%%%%%%%%%%%%%%%%%%%%%%%%%%%%%%%%%%%%%%%%%%%%%%%%%%%%%%%%%%%%%%
%%%%%%%%%%%%%%%%%%%%%%%%%%%%%%%%%%%%%%%%%%%%%%%%%%%%%%%%%%%%%%%%%%%%%%%%%%%%%%%%
\documentclass[12pt, a4paper]{article}



%%%%%%%%%%%%%%%%%%%%%%%%%%%%%%%%%%%%%%%%%%%%%%%%%%%%%%%%%%%%%%%%%%%%%%%%%%%%%%%%
% FONTS and ENCODING %%%%%%%%%%%%%%%%%%%%%%%%%%%%%%%%%%%%%%%%%%%%%%%%%%%%%%%%%%%
%%%%%%%%%%%%%%%%%%%%%%%%%%%%%%%%%%%%%%%%%%%%%%%%%%%%%%%%%%%%%%%%%%%%%%%%%%%%%%%%
%
% See: https://tex.stackexchange.com/q/59702
\usepackage[utf8]{inputenc}	      % allows input of accented characters directly from the keyboard
							              			% https://tex.stackexchange.com/questions/44694/fontenc-vs-inputenc
%
\usepackage{libertine}						% Linux Libertine (text) font family
\usepackage[libertine,						% math-mode fonts to harmonize with Libertine text fonts.
smallerops, varg, cmbraces, noamssymbols, ]{newtxmath}
                                  % [libertine] = substitutes libertine for Times
                                  % [smallerops] = smaller \sum,etc.
                                  % [varg] = distinctive math letters
                                  % [cmbraces] = thicker braces
                                  % [noamssymbols] = prevent TXAMSadd from loading
\let\Bbbk\relax										% disables newtxmath's definition of \Bbbk in favor of amssymb's (blackboard bold lowercase k)
                                  % This choice of disabling is arbitrary.
                                  % See: https://tex.stackexchange.com/a/587033
%\usepackage{newtxtext}		  		  % loads clones of reasonably well-match fonts (Helvetica and a monospaced)
%\usepackage{newpxtext,newpxmath} % based on Zapf's Palatino font
%
\usepackage{microtype}						% to break monospace (texttt) lines correctly
% the below does not work with ragged-right env. (leave disabled to stop warnings)
%\usepackage[htt]{hyphenat}       % to hyphenate monospace (texttt) lines correctly
							              			% See: https://tex.stackexchange.com/a/10378
\usepackage[T1]{fontenc} 	        % ensures proper hyphenation
																																	% why load after non-CM fonts?: https://tex.stackexchange.com/a/2869



%%%%%%%%%%%%%%%%%%%%%%%%%%%%%%%%%%%%%%%%%%%%%%%%%%%%%%%%%%%%%%%%%%%%%%%%%%%%%%%%
% AMS MATH %%%%%%%%%%%%%%%%%%%%%%%%%%%%%%%%%%%%%%%%%%%%%%%%%%%%%%%%%%%%%%%%%%%%%
%%%%%%%%%%%%%%%%%%%%%%%%%%%%%%%%%%%%%%%%%%%%%%%%%%%%%%%%%%%%%%%%%%%%%%%%%%%%%%%%
% \usepackage{amsmath}      % loads amstext, amsbsy, amsopn but not amssymb
                            % equation stuff (eqref, subequations, equation, align, gather, flalign, multline, alignat, split...)
                            % redundant by newtxmath
% \usepackage{amsfonts}     % may be redundant with amsmath
\usepackage{amssymb}        % may be redundant with amsmath (enable for \therefore)
% \numberwithin{equation}{section}  % reset equation counters at start of each "section" and prefix numbers by section number
% \numberwithin{figure}{section}    % reset figure   counters at start of each "section" and prefix numbers by section number



%%%%%%%%%%%%%%%%%%%%%%%%%%%%%%%%%%%%%%%%%%%%%%%%%%%%%%%%%%%%%%%%%%%%%%%%%%%%%%%%
% LAYOUT %%%%%%%%%%%%%%%%%%%%%%%%%%%%%%%%%%%%%%%%%%%%%%%%%%%%%%%%%%%%%%%%%%%%%%%
%%%%%%%%%%%%%%%%%%%%%%%%%%%%%%%%%%%%%%%%%%%%%%%%%%%%%%%%%%%%%%%%%%%%%%%%%%%%%%%%
%
% force ~71 (65) characters per line
% Credit: https://tex.stackexchange.com/a/59636
\usepackage{geometry}
\newlength{\alphabet}
\settowidth{\alphabet}{\normalfont abcdefghijklmnopqrstuvwxyz}
\geometry{textwidth = 2.75\alphabet}		% (Note: 2.75 * 26 = 71.5)
% wider for math-heavy, narrower for text-heavy
\makeatletter
\if@twoside
  \geometry{hmarginratio = {2:3}}
\else
  \geometry{hmarginratio = {1:1}}
\fi
\makeatother
%
%
\linespread{1.25}
%
% remove spaces around align (and other display math) envs
% Credit: https://tex.stackexchange.com/a/47403
%\usepackage{etoolbox}
%\newcommand{\zerodisplayskips}{%
%  \setlength{\abovedisplayskip}{12pt}%
%  \setlength{\belowdisplayskip}{12pt}%
%% "short skip" refers to when the text line above/below the display math
%% --ENDS (horizontally) before the display math BEGINS (horizontally)
%% --so the math is raised a bit (vertically) to not leave so much whitespace.
%% --See: https://tex.stackexchange.com/a/30913
%  \setlength{\abovedisplayshortskip}{5pt}%
%  \setlength{\belowdisplayshortskip}{5pt}}
%\appto{\normalsize}{\zerodisplayskips}
%\appto{\small}{\zerodisplayskips}
%\appto{\footnotesize}{\zerodisplayskips}
%



%%%%%%%%%%%%%%%%%%%%%%%%%%%%%%%%%%%%%%%%%%%%%%%%%%%%%%%%%%%%%%%%%%%%%%%%%%%%%%%%
% OTHER PACKAGES %%%%%%%%%%%%%%%%%%%%%%%%%%%%%%%%%%%%%%%%%%%%%%%%%%%%%%%%%%%%%%%
%%%%%%%%%%%%%%%%%%%%%%%%%%%%%%%%%%%%%%%%%%%%%%%%%%%%%%%%%%%%%%%%%%%%%%%%%%%%%%%%
\usepackage{enumitem}
\usepackage{mathtools}				% for multlined, dcases environment
\usepackage{lipsum}						% to fill in with arbitrary text
%\usepackage{datetime2}				% to display \currenttime in \date
%
\usepackage{ntheorem}
\newtheorem{theorem}{Theorem}
%
\usepackage{tcolorbox}				% currently, idk
\tcbuselibrary{most}					% currently, only for Highlight
\usepackage{fancybox}         % for text boxes such as those surrounding \Comments
%
\usepackage{derivative}			  % for upright "d"s and all sorts of derivatives
\usepackage{bm}										% bold math symbols (like vectors)

\usepackage{color}
\definecolor{warm}{HTML}{FFEEBB} % for a warm page background color



%%%%%%%%%%%%%%%%%%%%%%%%%%%%%%%%%%%%%%%%%%%%%%%%%%%%%%%%%%%%%%%%%%%%%%%%%%%%%%%%
% GRAPHICX %%%%%%%%%%%%%%%%%%%%%%%%%%%%%%%%%%%%%%%%%%%%%%%%%%%%%%%%%%%%%%%%%%%%%
%%%%%%%%%%%%%%%%%%%%%%%%%%%%%%%%%%%%%%%%%%%%%%%%%%%%%%%%%%%%%%%%%%%%%%%%%%%%%%%%
\usepackage{graphicx} 		            % options = [final]  = all graphics displayed, regardless of draft option in class
                                      % ^[final] does option clash
                                      % options = [pdftex] = necessary (?) if import PDF files
                                      % no option : when importing ps- and eps-files (?)
\usepackage[export]{adjustbox}
\graphicspath{ {./images/} }  	  		% tell LaTeX where to look for images
% \DeclareGraphicsExtensions{.pdf, .PDF, .jpg, .JPG, .jpeg, .JPEG, .png, .PNG, .bmp, .BMP, .eps, .ps}
\usepackage{float}				            % Improved interface for floating objects; add [H] option
\usepackage{subcaption}		            % allows subfigures (inside figure envs)



%%%%%%%%%%%%%%%%%%%%%%%%%%%%%%%%%%%%%%%%%%%%%%%%%%%%%%%%%%%%%%%%%%%%%%%%%%%%%%%%
% HYPERREF (last) then HYPCAP %%%%%%%%%%%%%%%%%%%%%%%%%%%%%%%%%%%%%%%%%%%%%%%%%%
%%%%%%%%%%%%%%%%%%%%%%%%%%%%%%%%%%%%%%%%%%%%%%%%%%%%%%%%%%%%%%%%%%%%%%%%%%%%%%%%
%
% See: https://tex.stackexchange.com/q/1863
\usepackage[
  pdflang       = {en-US},
  pdftex, 
  final,                      % if you DO want to have clickable-colorful links
  pdfstartview  = FitV,
  linktocpage   = false,      % ToC, LoF, LoT place hyperlink on page number, rather than entry text
  breaklinks    = true,       % so long urls are correctly broken across lines
  % pagebackref = false,      % add page number in bibliography and link to position in document where cited
]{hyperref}					          % hyperref should be last loaded package; see link above
%
% make better PDF
\hypersetup{
  pdftitle  = {CSCI-2725 Homework 3},
  pdfauthor = {Yitao Tian},
} % setup PDF
%\usepackage{tagpdf} % TK uncomment. add tags to content



%%%%%%%%%%%%%%%%%%%%%%%%%%%%%%%%%%%%%%%%%%%%%%%%%%%%%%%%%%%%%%%%%%%%%%%%%%%%%%%%
% CUSTOM MACROS %%%%%%%%%%%%%%%%%%%%%%%%%%%%%%%%%%%%%%%%%%%%%%%%%%%%%%%%%%%%%%%%
%%%%%%%%%%%%%%%%%%%%%%%%%%%%%%%%%%%%%%%%%%%%%%%%%%%%%%%%%%%%%%%%%%%%%%%%%%%%%%%%
% right-aligned (strong) qed symbol (for non-proof-environments)
\newcommand{\QQEEDD}{\tag*{$\blacksquare$}}
%
% right-aligned (weak) qed symbol (for non-proof-envs)
\newcommand{\QED}{\tag*{$\whitesquare$}}
%
% creates a phantom relational symbol (e.g. "=" (default) or "\leq")
% Purpose: split sums/differences, but with human-understandable spacing
% multiline/split expressions (after relational)
% must be horizontally to the right of the relational
% Credit: https://tex.stackexchange.com/a/448386
\newcommand*{\PhantomRel}[1][=]{\mathrel{\phantom{#1}}}
%
% word/phrase inline highlighting (for final answers).
% Color: green
% Credit: https://tex.stackexchange.com/a/234175 (less SLOW than tikz method on same page?)
\newtcbox{\Highlight}[1][]{enhanced, colframe = green, colback = green!15, 
       frame style = {opacity = 0.25}, interior style = {opacity = 0.25}, 
       nobeforeafter, tcbox raise base, shrink tight, extrude by = 1mm, #1}
%
% comment to explain step of proof
% Catalogue of boxes (e.g. ovalbox): https://tex.stackexchange.com/a/373420
\newcommand*{\Comment}[1]{\tag*{\ovalbox{\begin{tiny}
  \textrm{#1}
\end{tiny}}}}
%
% Bracketed array (i.e., to display numerical column/row vector, matrix)
% Credit: https://tex.stackexchange.com/a/156022
\newcommand{\ArrVect}[1]{\mspace{-2mu}\Bigl[\!
\begin{smallmatrix}
  #1
\end{smallmatrix}\!\Bigr]}
%
% bold-roman to represent vectors (as variables)
\newcommand*{\Vect}[1]{\bm{\mathrm{#1}}}
%
% evaluate (e.g. integrals, functions, derivatives)
% Credit: https://tex.stackexchange.com/a/15897
% #1: expression to be evaluated;
% #2: variable to be set; #3: lower bound; #4: upper bound
\newcommand{\Eval}[4]{\left.#1\mspace{2mu}\right\rvert_{\,#2\,=\,#3}^{\,#4}}
%
% "such that" bar in set-builder notation
\newcommand{\SuchThat}{\mathrel{}\middle|\mathrel{}}
%
% shorthand for \odif of `derivative` package
% varargs (csv) to match \odif's csv varargs
% See (csv varargs): https://tex.stackexchange.com/a/286717
\renewcommand*{\d}[1]{%
  \forcsvlist\odif{#1}}
%
% "del" operator for boundaries of spaces, from the `derivative` package
% of note: "style-notation=multiple" makes the variable (not "del") NOT SUBscript
% command's naming follows that of `derivative` package
\newcommand*{\del}[1]{\pdif[style-notation=multiple][#1]}
%
% (hyper-)volume differential, but "Vol" is romanized (unlike x or y (italicized))
\newcommand*{\dVol}{\odif{\textrm{Vol}}}
%
% absolute value and norm delimiters
% Credit: https://tex.stackexchange.com/a/43009
\DeclarePairedDelimiter\abs{\lvert}{\rvert}%
\DeclarePairedDelimiter\norm{\lVert}{\rVert}%
% Swap the definitions so that \abs,\norm resizes the brackets,
% and the starred (*) versions do not.
\makeatletter
\let\oldabs\abs
\def\abs{\@ifstar{\oldabs}{\oldabs*}}
\let\oldnorm\norm
\def\norm{\@ifstar{\oldnorm}{\oldnorm*}}
\makeatother
%
% R^n (isomorphic to) euclidean vector space.
% #1: number of dimensions
\newcommand*{\R}[1][n]{\bm{\mathrm{R}}^{#1}}
%
% redefine \int to have (on top of default) negative space (idk how much (mathopen{}))
% works with varargs bounds/limits
% Credit: https://tex.stackexchange.com/a/277428
\makeatletter
\let\int@original\int
\def\int{\int@checkfirstsb}
\def\int@checkfirstsb{\@ifnextchar_{\int@checksecondsp}{\int@checkfirstsp}}
\def\int@checkfirstsp{\@ifnextchar^{\int@checksecondsb}{\int@{}{}}}
\def\int@checksecondsp_#1{\@ifnextchar^{\int@grabsp{#1}}{\int@{#1}{}}}
\def\int@checksecondsb^#1{\@ifnextchar_{\int@grabsb{#1}}{\int@{}{#1}}}
\def\int@grabsb#1_#2{\int@{#2}{#1}}
\def\int@grabsp#1^#2{\int@{#1}{#2}}
\def\int@#1#2{\int@original\ifblank{#1}{}{_{#1}}\ifblank{#2}{}{^{#2}}\mathopen{}} % redefinition
\makeatother



%%%%%%%%%%%%%%%%%%%%%%%%%%%%%%%%%%%%%%%%%%%%%%%%%%%%%%%%%%%%%%%%%%%%%%%%%%%%%%%%
% CUSTOM ENVIRONMENTS %%%%%%%%%%%%%%%%%%%%%%%%%%%%%%%%%%%%%%%%%%%%%%%%%%%%%%%%%%
%%%%%%%%%%%%%%%%%%%%%%%%%%%%%%%%%%%%%%%%%%%%%%%%%%%%%%%%%%%%%%%%%%%%%%%%%%%%%%%%
%
% Shifrin problem enumerate
\newlist{Shifrin}{enumerate}{2}         % Custom 2-deep enumerate list
\setlist[Shifrin, 1]{%                  % Configure the behaviour of level 1 entries
    label				= \arabic{Shifrini}.,   % 1., 2., 3., ...
    leftmargin	= \parindent,
    rightmargin	= 10pt
}
\setlist[Shifrin, 2]{%                  % Configure the behaviour of level 2 entries
    label				= (\arabic{Shifrinii}), % e.g. (1), (2), (3), ...
    leftmargin	= 15pt,
    rightmargin	= 15pt
}

% REPLACE ALL OF ABOVE
\usepackage{MintedUnbreakableCodeBlock}
\usepackage[normalem]{ulem} % for striking-through text (don't remove option)
%\setlength{\textfloatsep}{1\baselineskip plus 0.2\baselineskip minus 0.2\baselineskip} % Credit: https://tex.stackexchange.com/a/60479
\newcommand{\squeezeup}{\vspace{-16pt}}% Credit: https://tex.stackexchange.com/a/60488




%%%%%%%%%%%%%%%%%%%%%%%%%%%%%%%%%%%%%%%%%%%%%%%%%%%%%%%%%%%%%%%%%%%%%%%%%%%%%%%
% TITLE PAGE %%%%%%%%%%%%%%%%%%%%%%%%%%%%%%%%%%%%%%%%%%%%%%%%%%%%%%%%%%%%%%%%%%
%%%%%%%%%%%%%%%%%%%%%%%%%%%%%%%%%%%%%%%%%%%%%%%%%%%%%%%%%%%%%%%%%%%%%%%%%%%%%%%
\title{CSCI-2725 Homework 3}
\author{Yitao Tian}
\date{March 17, 2023}



%%%%%%%%%%%%%%%%%%%%%%%%%%%%%%%%%%%%%%%%%%%%%%%%%%%%%%%%%%%%%%%%%%%%%%%%%%%%%%%
% THE ACTUAL DOCUMENT %%%%%%%%%%%%%%%%%%%%%%%%%%%%%%%%%%%%%%%%%%%%%%%%%%%%%%%%%
%%%%%%%%%%%%%%%%%%%%%%%%%%%%%%%%%%%%%%%%%%%%%%%%%%%%%%%%%%%%%%%%%%%%%%%%%%%%%%%
\begin{document}

\maketitle

\textbf{Extra credit:} There is 5 percentage extra credit if you don't submit hand-written homework (including the diagrams).
You can use \LaTeX\ or any other tool to write your homework.

\begin{enumerate}

    \item (4 points each) \textbf{State} the following as true or false.
    \begin{enumerate}
    
        \item Recursive functions often have fewer local variables than the equivalent iterative functions. \\
        \hspace*{\fill} \textbf{Answer:} \Highlight{True}; though more intermediate parameters.

        \item Recursive functions must contain a path that does not contain recursive call. \\
        \hspace*{\fill} \textbf{Answer:} \Highlight{True}; this path is called the ``base case''.
        
        \item Recursive functions are always less efficient than the corresponding iterative functions in terms of Big-$O$ complexity. \\
        \hspace*{\fill} \textbf{Answer:} \Highlight{False}; not always, but typically.
    
    \end{enumerate}



    \newpage


    
    \item (6 points each)
    \begin{MintedUnbreakableCodeBlock}{java}
        int fun(int num) {
            if (num < 2) {
                return 1;
            }
            else if (num % 2 == 0) {
                return fun(num / 2);
            }
            else {
                return fun(3 * num + 1);
            }
        }\end{MintedUnbreakableCodeBlock}
       
    \begin{enumerate}
        
        \item How many recursive calls will be made for function call \verb|fun(7)|?
            Show your work on how you got the number of recursive calls. \\
            \hspace*{\fill} \textbf{Answer:} \Highlight{Seventeen (17)} function calls including first call (\verb|fun(7)|). \\
            \textbf{Justification:} fun(7), fun(22), fun(11), fun(34), fun(17), fun(52), fun(26), fun(13), fun(40), fun(20), fun(10), fun(5), fun(16), fun(8), fun(4), fun(2), fun(1)
        
        \item How many recursive calls will be made for function call \verb|fun(8)|?
            Show your work on how you got the number of recursive call.  \\
            \hspace*{\fill} \textbf{Answer:} \Highlight{Four (4)} function calls including first call (\verb|fun(8)|). \\
            \textbf{Justification:} fun(8), fun(4), fun(2), fun(1)

        \item Do you see any problem in the above function?
            If so, give justification to your answer. \\
            \hspace*{\fill} \textbf{Answer:} \Highlight{Yes}, there is a problem; the function should not have wildly different complexity for its inputs.
    
    \end{enumerate}



    \newpage

    

    \item (6 points each) What are the outputs of the following function calls?
        Show your work to justify your answer.
    \begin{MintedUnbreakableCodeBlock}{java}
        int fun(int k, int n)
        {
            if (n == k) { return k; }
            else {
                if (n > k) { return fun(k, n - k); }
                else { return fun(k - n, n); }
            }
        }\end{MintedUnbreakableCodeBlock}

    \begin{enumerate}
        
        \item \verb|System.out.print(fun(10,6));| \\
        \hspace*{\fill} \textbf{Answer:} \Highlight{Prints 2} \\
        \textbf{Justification:} fun(10,6), fun(4,6), fun(4,2), fun(2,2)

        \item \verb|System.out.print(fun(7,13));| \\
        \hspace*{\fill} \textbf{Answer:} \Highlight{Prints 1} \\
        \textbf{Justification:} fun(7,13), fun(7,6), fun(1,6), fun(1,5), fun(1,4), fun(1,3), fun(1,2), fun(1,1)

    \end{enumerate}



    \newpage



    \item (6 points) What is the output of the following function call?
        Show your work to justify your answer.
    \begin{MintedUnbreakableCodeBlock}{java}
        void Quiz( int n ) {
            if (n > 0) {
                System.out.print("0");
                Quiz(n - 1);
                System.out.print("1");
                Quiz(n - 1);
            }
        }\end{MintedUnbreakableCodeBlock}

    \sout{\texttt{System.out.println(Quiz(2));}} \verb|// not valid Java code| \\
    \verb|Quiz(2);| \\
    \hspace*{\fill} \textbf{Answer:} \Highlight{0, 0, 1, 1, 0, 1} \\
    \textbf{Justification:} Quiz(2),
        print 0,
        Quiz(1),
            print 0,
            Quiz(0),
            print 1,
            Quiz(0),
        print 1,
        Quiz(1),
            print 0,
            Quiz(0),
            print 1,
            Quiz(0)



    \newpage



    \item (4 points each) \textbf{Consider} the binary search tree below
        (the numbers inside the nodes are not the actual data; they are just numbers labeling the nodes).
    \begin{figure}[H]
        \includegraphics[max width=\textwidth/2, center]{"5.png"}
        \caption{A binary search tree}
        \label{fig:5-BST}
    \end{figure}\squeezeup

    \begin{enumerate}

        \item If node 1 is deleted, what node or nodes can replace node 1 in the tree? \\
        \hspace*{\fill} \textbf{Answer:} \Highlight{Nodes 5 and 6} can replace; they are the predecessor and successor, respectively.
        
        \item 4 2 7 5 1 6 8 3 is a traversal of the tree in which order? \\
        \hspace*{\fill} \textbf{Answer:} In \Highlight{"in-order"} order.
        
        \item 1 2 4 5 7 3 6 8 is a traversal of the tree in which order? \\
        \hspace*{\fill} \textbf{Answer:} In \Highlight{"pre-order"} order.
    
    \end{enumerate}



    \newpage



    \item (4 points each) \textbf{Consider} the following set of input data:
    \begin{equation*}
        50\ 72\ 96\ 94\ 100\ 26\ 13\ 11\ 9\ 2\ 10\ 14\ 25\ 51\ 33\ 16\ 95\ 12
    \end{equation*}
    
    \begin{enumerate}

        \item \textbf{Draw} a binary search tree using the above input (in the order given).
        \begin{figure}[H]
            \includegraphics[max width=\textwidth/2, center]{"6(a).png"}
            \caption{A BST based on the given, ordered input}
            \label{fig:6(a)-BST}
        \end{figure}\squeezeup
    
        \item What nodes are in level 3 of the tree? \\
        \hspace*{\fill} \textbf{Answer:} \Highlight{11, 14, 94, 100} (assuming root is level 0).
        
        \item \textbf{List} all the leaf nodes in the tree. \\
        \hspace*{\fill} \textbf{Answer:} \Highlight{2, 10, 12, 16, 33, 51, 95, 100}.
        
        \item What is the height of the binary search tree containing this input? \\
        \hspace*{\fill} \textbf{Answer:} \Highlight{5 excluding the root}, 6 including.
        
        \item \textbf{Give} the output of post-order traversal function called on the above tree. \\
        \hspace*{\fill} \textbf{Answer:} \Highlight{2 10 9 12 11 16 25 14 13 33 26 51 95 94 100 96 72 50}
        
        \item What is the maximum height of the binary search tree containing this input? \\
        \hspace*{\fill} \textbf{Answer:} \Highlight{17 excluding the root}, 18 including.

        \item What is the minimum height of a binary search tree containing this input? \\
        \hspace*{\fill} \textbf{Answer:} \Highlight{4 excluding the root}, 5 including.

        \item \textbf{Draw} the tree after deleting the node containing 94.
        \begin{figure}[H]
            \includegraphics[max width=\textwidth/2, center]{"6(h).png"}
            \caption{The BST after deleting 94}
            \label{fig:6(h)-BST}
        \end{figure}\squeezeup

        \newpage

        \item \textbf{Use} the above tree (from part (h)) and draw the tree after deleting the node containing 13.
        \begin{figure}[H]
            \includegraphics[max width=\textwidth/2, center]{"6(i).png"}
            \caption{The BST after deleting 13}
            \label{fig:6(i)-BST}
        \end{figure}\squeezeup

        \item \textbf{Use} the above tree (from part (i)) and draw the tree after deleting the node containing 50.
        \begin{figure}[H]
            \includegraphics[max width=\textwidth/2, center]{"6(j).png"}
            \caption{The BST after deleting 50}
            \label{fig:6(j)-BST}
        \end{figure}\squeezeup

    \end{enumerate}

\end{enumerate}

\end{document}


50, 72, 96, 94, 100, 26, 13, 11, 9, 2, 10, 14, 25, 51, 33, 16, 95, 12
2, 9, 10, 11, 12, 13, 14, 16, 25, 26, 33, 50, 51, 72, 94, 95, 96, 100

% TK TODO
% X fix all off-by-one height problems
% X try to run code snippets in python (with printing added) to verify results
% - add boxes around figures
% - organize function call csv